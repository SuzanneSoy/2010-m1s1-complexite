\documentclass{article}

\usepackage[utf8]{inputenc}
\usepackage[T1]{fontenc}
\usepackage[frenchb]{babel}
\usepackage{tikz}
\usetikzlibrary{chains,positioning,matrix}

\begin{document}

\section{Partie théorique}
\subsection{Partie algorithmique}

\begin{figure}[h!]
  \centering
  \begin{tikzpicture}[node distance=3cm]
    \node (J1) {$J_{1}$};
    \node (J2) [above of=J1] {$J_{2}$};
    \node (J3) [right of=J1] {$J_{3}$};
    \draw[->] (J1) -- (J2);
    \draw[->] (J1) -- (J3);
    \draw[->] (J3) -- (J2);
  \end{tikzpicture}
  \caption{Graphe G}
  \label{fig:graphe-g}
\end{figure}

\begin{figure}[h!]
  \centering
  \colorlet{affectation}{green!50!black}
  \colorlet{auxiliaire}{black}
  \colorlet{précédence}{blue}
  \begin{tikzpicture}[
      affectation/.style = {
        draw=affectation,
        ->
      },
      auxiliaire/.style = {
        draw=auxiliaire,
        ->
      },
      précédence/.style = {
        draw=précédence,
        ->
      },
      capacité/.style = {
        fill=white,
        font=\footnotesize
      },
      capacité affectation/.style = {
        text=affectation,
        capacité
      },
      capacité auxiliaire/.style = {
        text=auxiliaire,
        capacité
      },
      capacité précédence/.style = {
        text=précédence,
        capacité
      }
    ]
    
    \matrix[matrix of math nodes, nodes in empty cells, row sep=1cm, column sep=1cm] (m) {
      & v_{1,0} & v_{1,1} & v_{1,2} & v_{1,3} & v_{1,4} & v_{1,5} & \\
      s & v_{3,0} & v_{3,1} & v_{3,2} & v_{3,3} & v_{3,4} & v_{3,5} & t \\
      & v_{2,0} & v_{2,1} & v_{2,2} & v_{2,3} & v_{2,4} & & \\
    };
    
    %% Penser a rajouter les J1, J2 et J3 a gauche du graphe.
    
  	\draw[affectation] (m-1-2)-- node[capacité affectation]{0} (m-1-3);
	\draw[affectation] (m-1-3)-- node[capacité affectation]{2} (m-1-4);
	\draw[affectation] (m-1-4)-- node[capacité affectation]{5} (m-1-5);
	\draw[affectation] (m-1-5)-- node[capacité affectation]{0} (m-1-6);
	\draw[affectation] (m-1-6)-- node[capacité affectation]{1} (m-1-7);

	\draw[affectation] (m-2-2)-- node[capacité affectation]{1} (m-2-3);
	\draw[affectation] (m-2-3)-- node[capacité affectation]{10} (m-2-4);
	\draw[affectation] (m-2-4)-- node[capacité affectation]{2} (m-2-5);
	\draw[affectation] (m-2-5)-- node[capacité affectation]{3} (m-2-6);
	\draw[affectation] (m-2-6)-- node[capacité affectation]{3} (m-2-7);

	\draw[affectation] (m-3-2)-- node[capacité affectation]{1} (m-3-3);
	\draw[affectation] (m-3-3)-- node[capacité affectation]{1} (m-3-4);
	\draw[affectation] (m-3-4)-- node[capacité affectation]{2} (m-3-5);
	\draw[affectation] (m-3-5)-- node[capacité affectation]{4} (m-3-6);
	
  	\draw[auxiliaire] (m-2-1)-- node[capacité auxiliaire]{$\infty$} (m-1-2);
  	\draw[auxiliaire] (m-2-1)-- node[capacité auxiliaire]{$\infty$} (m-2-2);
  	\draw[auxiliaire] (m-2-1)-- node[capacité auxiliaire]{$\infty$} (m-3-2);
  	\draw[auxiliaire] (m-1-7)-- node[capacité auxiliaire]{$\infty$} (m-2-8);
  	\draw[auxiliaire] (m-2-7)-- node[capacité auxiliaire]{$\infty$} (m-2-8);
  	\draw[auxiliaire] (m-3-6)--(m-3-7.center)-- node[capacité auxiliaire]{$\infty$} (m-2-8);
	
	\draw[précédence] (m-1-2)-- node[capacité précédence]{$\infty$} (m-2-3);
	\draw[précédence] (m-1-3)-- node[capacité précédence]{$\infty$} (m-2-4);
	\draw[précédence] (m-1-4)-- node[capacité précédence]{$\infty$} (m-2-5);
	\draw[précédence] (m-1-5)-- node[capacité précédence]{$\infty$} (m-2-6);
	\draw[précédence] (m-1-6)-- node[capacité précédence]{$\infty$} (m-2-7);
    
	\draw[précédence] (m-2-2)-- node[capacité précédence]{$\infty$} (m-3-3);
	\draw[précédence] (m-2-3)-- node[capacité précédence]{$\infty$} (m-3-4);
	\draw[précédence] (m-2-4)-- node[capacité précédence]{$\infty$} (m-3-5);
	\draw[précédence] (m-2-5)-- node[capacité précédence]{$\infty$} (m-3-6);

	\draw[précédence] (m-1-2)-- node[capacité précédence,pos=0.3]{$\infty$} (m-3-3);
	\draw[précédence] (m-1-3)-- node[capacité précédence,pos=0.3]{$\infty$} (m-3-4);
	\draw[précédence] (m-1-4)-- node[capacité précédence,pos=0.3]{$\infty$} (m-3-5);
	\draw[précédence] (m-1-5)-- node[capacité précédence,pos=0.3]{$\infty$} (m-3-6);

  \end{tikzpicture}
  \caption{Graphe G*}
  \label{fig:graphe-g*}
\end{figure}

\subsection{Demonstration}

Démonstration par construction :
On éffectue un tri topologique sur le graphe des contraintes de précédence, le graphe resultant est $G' (\{J_{1}\ldots J_{n}\}, E') $ on a donc :
$$\forall J_{i} \quad \not\exists \ j < i \quad | \quad \exists (J_{j}, J_{i}) \in E'$$
On transforme ensuite $G'$ en un graphe de flots avec l'algorithme donné dans le sujet.
Considérons les arcs entre les $v_{i,j}$ : 
\begin{itemize}
	\item Arcs d'affectation : ces arcs sont entre des sommets $v_{i,j}$ et $v_{k,l}$ avec $i = k$
	\item Arcs de précédences : ces arcs sont entre des sommets $v_{i,j}$ et $v_{k,l}$ avec $i < k$ à cause du tri topologique.
	\item Arcs auxiliaires : ces arcs ne sont pas entre des sommets $v_{i,j}$.
\end{itemize}
On va créer une $(s-t)-\mathrm{coupe}$ minimale. Comme cette coupe et minimale, aucun arc de capacité infinie n'a son origine dans $S$ et son extremité dans $\overline{S}$.
\end{document}
