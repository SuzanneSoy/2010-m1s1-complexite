\documentclass{article}

\usepackage[utf8]{inputenc}
\usepackage[T1]{fontenc}
\usepackage[frenchb]{babel}
\usepackage{tikz}
\usepackage{amsmath}
\usetikzlibrary{chains,positioning,matrix}

\begin{document}

\section{Partie théorique}
\subsection{Partie algorithmique}

\begin{figure}[h!]
  \centering
  \begin{tikzpicture}[node distance=3cm]
    \node (J1) {$J_{1}$};
    \node (J2) [above of=J1] {$J_{2}$};
    \node (J3) [right of=J1] {$J_{3}$};
    \draw[->] (J1) -- (J2);
    \draw[->] (J1) -- (J3);
    \draw[->] (J3) -- (J2);
  \end{tikzpicture}
  \caption{Graphe G}
  \label{fig:graphe-g}
\end{figure}

\begin{figure}[h!]
  \centering
  \colorlet{affectation}{green!50!black}
  \colorlet{auxiliaire}{black}
  \colorlet{précédence}{blue}
  \begin{tikzpicture}[
      affectation/.style = {
        draw=affectation,
        ->
      },
      auxiliaire/.style = {
        draw=auxiliaire,
        ->
      },
      précédence/.style = {
        draw=précédence,
        ->
      },
      capacité/.style = {
        fill=white,
        font=\footnotesize
      },
      capacité affectation/.style = {
        text=affectation,
        capacité
      },
      capacité auxiliaire/.style = {
        text=auxiliaire,
        capacité
      },
      capacité précédence/.style = {
        text=précédence,
        capacité
      }
    ]
    
    \matrix[matrix of math nodes, nodes in empty cells, row sep=1cm, column sep=1cm] (m) {
      & v_{1,0} & v_{1,1} & v_{1,2} & v_{1,3} & v_{1,4} & v_{1,5} & \\
      s & v_{3,0} & v_{3,1} & v_{3,2} & v_{3,3} & v_{3,4} & v_{3,5} & t \\
      & v_{2,0} & v_{2,1} & v_{2,2} & v_{2,3} & v_{2,4} & & \\
    };
    
    %% Penser a rajouter les J1, J2 et J3 a gauche du graphe.
    
  	\draw[affectation] (m-1-2)-- node[capacité affectation]{0} (m-1-3);
	\draw[affectation] (m-1-3)-- node[capacité affectation]{2} (m-1-4);
	\draw[affectation] (m-1-4)-- node[capacité affectation]{5} (m-1-5);
	\draw[affectation] (m-1-5)-- node[capacité affectation]{0} (m-1-6);
	\draw[affectation] (m-1-6)-- node[capacité affectation]{1} (m-1-7);

	\draw[affectation] (m-2-2)-- node[capacité affectation]{1} (m-2-3);
	\draw[affectation] (m-2-3)-- node[capacité affectation]{10} (m-2-4);
	\draw[affectation] (m-2-4)-- node[capacité affectation]{2} (m-2-5);
	\draw[affectation] (m-2-5)-- node[capacité affectation]{3} (m-2-6);
	\draw[affectation] (m-2-6)-- node[capacité affectation]{3} (m-2-7);

	\draw[affectation] (m-3-2)-- node[capacité affectation]{1} (m-3-3);
	\draw[affectation] (m-3-3)-- node[capacité affectation]{1} (m-3-4);
	\draw[affectation] (m-3-4)-- node[capacité affectation]{2} (m-3-5);
	\draw[affectation] (m-3-5)-- node[capacité affectation]{4} (m-3-6);
	
  	\draw[auxiliaire] (m-2-1)-- node[capacité auxiliaire]{$\infty$} (m-1-2);
  	\draw[auxiliaire] (m-2-1)-- node[capacité auxiliaire]{$\infty$} (m-2-2);
  	\draw[auxiliaire] (m-2-1)-- node[capacité auxiliaire]{$\infty$} (m-3-2);
  	\draw[auxiliaire] (m-1-7)-- node[capacité auxiliaire]{$\infty$} (m-2-8);
  	\draw[auxiliaire] (m-2-7)-- node[capacité auxiliaire]{$\infty$} (m-2-8);
  	\draw[auxiliaire] (m-3-6)--(m-3-7.center)-- node[capacité auxiliaire]{$\infty$} (m-2-8);
	
	\draw[précédence] (m-1-2)-- node[capacité précédence]{$\infty$} (m-2-3);
	\draw[précédence] (m-1-3)-- node[capacité précédence]{$\infty$} (m-2-4);
	\draw[précédence] (m-1-4)-- node[capacité précédence]{$\infty$} (m-2-5);
	\draw[précédence] (m-1-5)-- node[capacité précédence]{$\infty$} (m-2-6);
	\draw[précédence] (m-1-6)-- node[capacité précédence]{$\infty$} (m-2-7);
    
	\draw[précédence] (m-2-2)-- node[capacité précédence]{$\infty$} (m-3-3);
	\draw[précédence] (m-2-3)-- node[capacité précédence]{$\infty$} (m-3-4);
	\draw[précédence] (m-2-4)-- node[capacité précédence]{$\infty$} (m-3-5);
	\draw[précédence] (m-2-5)-- node[capacité précédence]{$\infty$} (m-3-6);

	\draw[précédence] (m-1-2)-- node[capacité précédence,pos=0.3]{$\infty$} (m-3-3);
	\draw[précédence] (m-1-3)-- node[capacité précédence,pos=0.3]{$\infty$} (m-3-4);
	\draw[précédence] (m-1-4)-- node[capacité précédence,pos=0.3]{$\infty$} (m-3-5);
	\draw[précédence] (m-1-5)-- node[capacité précédence,pos=0.3]{$\infty$} (m-3-6);

  \end{tikzpicture}
  \caption{Graphe G*}
  \label{fig:graphe-g*}
\end{figure}

\subsubsection{Question 2}

Démonstration par construction :
On éffectue un tri topologique sur le graphe des contraintes de précédence, le graphe resultant est $G' (\{J_{1}\ldots J_{n}\}, E') $ on a donc :
$$\forall J_{i} \quad \not\exists \ j < i \quad | \quad \exists (J_{j}, J_{i}) \in E'$$
On transforme ensuite $G'$ en un graphe de flots avec l'algorithme donné dans le sujet.
Considérons les arcs entre les $v_{i,j}$ : 
\begin{itemize}
	\item Arcs d'affectation : ces arcs sont entre des sommets $v_{i,j}$ et $v_{k,l}$ avec $i = k$
	\item Arcs de précédences : ces arcs sont entre des sommets $v_{i,j}$ et $v_{k,l}$ avec $i < k$ à cause du tri topologique.
	\item Arcs auxiliaires : ces arcs ne sont pas entre des sommets $v_{i,j}$.
\end{itemize}
On va créer une $(s-t)-\mathrm{coupe}$ minimale. Comme cette coupe et minimale, aucun arc de capacité infinie n'a son origine dans $S$ et son extremité dans $\overline{S}$.

TODO : on montre que si c'est une coupe min, il ne sort qu'un et un
seul arc d'affectation par job. Il faut aussi montrer qu'il existe.

\subsubsection{Question 3}
TODO : Attention à la phrase suivante, ce n'est pas tout à fait ce qu'on a montré dans l'exo 2

Dans l'exercice précédent, on a montré que de toute coupe minimale sort un et un seul art d'affectation par job.

On cherche à associer un ordonancement réalisable à toute coupe minimale.

On va construire cet ordonancement de la manière suivante : à chaque
fois qu'un arc d'affectation $v_{i,t}, v_{i,t+1}$ traverse la coupe,
on exécute le job $i$ à l'instant $t$ dans l'ordonancement.

On cherche un ordonancement, donc une suite de paires
$(\text{tâche},\text{date de début})$ respectant les dépendances,
autrement dit chaque tâche apparaît après ses dépendances dans la
suite.

Comme chaque arc de précédence a une capacité finie, pour que la coupe
soit minimale, cela signifie qu'aucun arc de précédence ne sort de la
coupe. Cela signifie donc qu'à chaque fois qu'on exécute un job (à
chaque fois qu'un arc d'affectation sort de la coupe), tous les arcs
de précédence entrants dans ce noeud ont leur extrémité déjà présente
dans la partie \og gauche \fg de la coupe (TODO : on n'a pas prouvé
ça, on a prouvé pour tous les sortants, mais pas les entrants. Il faut
montrer qu'en partant de la source, on est obligié d'avoir tous les
arcs de précédence entrants.), et donc toutes les tâches dont dépend
celle-là ont été exécutées à un temps antérieur.

On cherche un ordonancement réalisable, c'est à dire pour lequel
toutes les tâches peuvent être menées à bout durant le temps
imparti.

La propriété énoncée dans l'exercice 2 nous indique que dans toute
coupe minimale, un et un seul arc d'affectation par job sort de la
coupe. Cela signifie que chaque job a commencé à être exécuté. Comme
il exite un noeud pour je job $j$ à l'instant $t$ si et seulement s'il
y a le temps de l'exécuter (\ogà chaque date de début possible\fg),
cela signifie que tous les jobs commencés ont eu le temps d'être
terminés, et comme nous venons de voir que tous les jobs ont pu être
commencés, ils ont tous pu être terminés. Donc l'ordonancement est
réalisable.

\subsubsection{Question 4}

On construit la coupe à partir de l'ordonancement de la même manière
qu'on a construit l'ordonancement à partir de la coupe dans l'exercice
précédent, mais en suivant l'algorithme dans l'autre sens~:

Si on exécute le job $i$ à l'instant $t$ dans l'ordonancement, alors
tous les noeuds $v_{i,t'}$ avec $t' \leq t$ sont dans la partie \og
gauche\fg de la coupe. De plus, s appartient lui aussi à la partie
gauche de la coupe.

TODO : et les arcs de précédence ? prouver qu'aucun ne sort de la
coupe dans notre construction.

La capacité de cette coupe est la somme de la capacité de tous les
arcs qui sortent de la coupe, autrement dit la somme des capacités des
arcs $v_{i,t}, v_{i,t+1}$. Comme la capacité de ces arcs est égale au
coût d'exécution de la tâche i à l'instant t, on a bien égalité entre
la somme des capacités et la somme des coûts de démarrage des jobs,
donc la capacité de la coupe est égale au coût de l'ordonancement.

\subsection{Exercice 2}

Il existe un ensemble de chemins d'arcs disjoints de cardinal 3~:

$$
\begin{array}{ccccccc}
  1 & \rightarrow & 2 & \rightarrow & 3 & \rightarrow & 6 \\
  1 & \rightarrow & 3 & \rightarrow & 4 & \rightarrow & 6 \\
  1 & \rightarrow & 4 & \rightarrow & 5 & \rightarrow & 6 \\
\end{array}
$$

Cherchons s'il en existe un de cardinal 4. Voici la liste des chemins,
obtenue par un parcours en profondeur (en prennant toujours en premier
les sommets voisins avec le numéro le plus petit possible) :

TODO : numéroter les "équations"
$$
\begin{array}{ccccccccccc}
  1 & \rightarrow & 2 & \rightarrow & 3 & \rightarrow & 4 & \rightarrow & 5 & \rightarrow & 6 \\
  1 & \rightarrow & 2 & \rightarrow & 3 & \rightarrow & 4 & \rightarrow & 6 &             &   \\
  1 & \rightarrow & 2 & \rightarrow & 3 & \rightarrow & 6 &             &   &             &   \\
  1 & \rightarrow & 3 & \rightarrow & 4 & \rightarrow & 5 & \rightarrow & 6 &             &   \\
  1 & \rightarrow & 3 & \rightarrow & 4 & \rightarrow & 6 &             &   &             &   \\
  1 & \rightarrow & 3 & \rightarrow & 6 &             &   &             &   &             &   \\
  1 & \rightarrow & 4 & \rightarrow & 5 & \rightarrow & 6 &             &   &             &   \\
  1 & \rightarrow & 4 & \rightarrow & 6 &             &   &             &   &             &   \\
\end{array}
$$

Voyons les ensembles qui contiennent le chemin $A$~: On ne peut pas
prendre les chemins $B$ et $C$, car ils ont l'arc $(1,2)$ en commun
avec $A$. On ne peut pas prendre non plus les chemin $D$ et $G$, car
ils ont l'arc $(5,6)$ en commun avec $A$, ni le chemin $E$ à cause de
l'arc $(3,4)$. Il ne reste plus que les chemins $F$ et $H$ qu'on
pourrait peut-être prendre si on prend $A$, mais le cardinal de
l'ensemble serait alors 3, donc on n'améliorerait pas le résultat
existant, et donc ce n'est pas la peine de chercher si ces chemins
sont \og compatibles\fg avec $A$.

Procédons de la même manière pour $B$ (sachant que $A$ ne peut pas
faire partie de l'ensemble). Si on a le chemin $B$, alors on ne peut
pas avoir~:
\begin{itemize}
  \item $C$ (arc $(2,3)$),
  \item $D$ (arc $(3,4)$),
  \item $E$ (arc $(3,4)$),
  \item $H$ (arc $(4,6)$).
\end{itemize}
À partir de ce moment, il ne reste plus que $F$ et $G$, l'ensemble
serait de cardinal 3, donc $B$ ne peut pas être dans un ensemble de
cardinal 4.

Passons à $D$ avec $A$ et $B$ exclus. On ne peut pas avoir~:
\begin{itemize}
\item $E$ (arc $(3,4)$),
\item $F$ (arc $(1,3)$),
\item $G$ (arc $(4,5)$).
\end{itemize}
Donc $D$ n'est pas dans l'ensemble.

Passons à $F$ avec $A,B,D$ exclus. On ne peut pas avoir~:
\begin{itemize}
\item $C$ (arc $(3,6)$),
\item $E$ (arc $(1,3)$).
\end{itemize}
Donc $F$ n'est pas dans l'ensemble.

Comme $A,B,D,F$ ne sont pas dans l'ensemble, et que nous avons
seulement 8 candidats, la seule possibilité qui reste pour un ensemble
de cardinal 4 est $C,E,G,H$, or dans cet ensemble, l'arrête $(1,4)$
est commune à $G$ et $H$, donc on ne peut pas construire un ensemble
de chemins d'arcs disjoints de taille 4 (donc pas de taille supérieure
à 4 non plus).

Conclusion : Le nombre maximum de chemins d'arcs disjoints est 3.


\end{document}