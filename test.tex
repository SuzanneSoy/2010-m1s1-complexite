\documentclass{article}

\usepackage[utf8]{inputenc}
\usepackage[T1]{fontenc}
\usepackage[frenchb]{babel}
\usepackage{tikz}
\usepackage{amsmath}
\usepackage{listings}
\usepackage{amssymb}
\usetikzlibrary{chains,positioning,matrix,arrows,decorations,calc}
\title{Rapport de projet : FMIN105\\ Cours algorithmique / complexité / calculabilité}
\author{\textsc{Bonavero} Yoann \\ \textsc{Brun} Bertrand \\ \textsc{Charron} John \\ \textsc{Dupéron} Georges}
\date{}

\setlength{\parindent}{0pt}
\setlength{\parskip}{2ex}

\newcounter{exocount}
\setcounter{exocount}{0}

\newcounter{enoncecount}
\setcounter{enoncecount}{0}

\newenvironment{enonce}
{
\stepcounter{enoncecount}
\bf\small \arabic{enoncecount}.
\begin{bf}
}
{
\end{bf}
}


\newcounter{sousenoncecount}
\setcounter{sousenoncecount}{0}
\newenvironment{sousenonce}
{
\stepcounter{sousenoncecount}
\bf\small (\alph{sousenoncecount})
\begin{bf}
}
{
\end{bf}
}

\begin{document}
\setcounter{page}{0}
\pagestyle{plain}

\begin{figure}[h!]
  %% «Paramètres»
  \xdef\maxdots{23}
  \xdef\maxdotsX{5}
  \xdef\maxdotsY{5}
  \xdef\maxnums{20}
  \xdef\maxcoordsX{3}
  \xdef\maxcoordsY{3}
  
  %% ===== Code is below =====
  
  %% Macros crades
  \makeatletter
  \def\myadv#1#2{%
    \xdef\incr@backup{\the\@tempcnta}%
    \@tempcnta=#1{}%
    \advance\@tempcnta by #2{}%
    \xdef#1{\the\@tempcnta}%
    \@tempcnta=\incr@backup{}%
  }
  \def\incr#1{%
    \myadv#1{1}%
  }
  \def\decr#1{%
    \myadv#1{-1}%
  }
  \makeatother
  \centering
  \begin{tikzpicture}[
    scale=1.4,
    dot/.style={
      circle,
      inner sep=0.7pt,
      fill=black
    },
    dotphantom/.style={
      circle,
      inner sep=3pt,
      fill=none,
      draw=none
    },
    numero/.style={rectangle, fill=white, inner sep=2pt, text=red!50!black},
    coord/.style={darkgray, scale=.8},
    %
    bigarrow/.style={
      ->,
      thick,
      draw=green!50!black
    },
    constatation1/.style={
      circle,
      inner sep=1.5pt,
      outer sep=0.8pt,
      fill=white,
      text=red,
      anchor=south,
      rotate around={45:(0,0)}
    },
    constatation2/.style={
      text=green!50!black,
      node distance=0.6cm
    }
    ]
    
    % %% Digonale
    % \draw[bigarrow,red] (0,0) -- (.5*7.5,-.5*7.5);
    
    % Définitions pour le code ci-dessous.
    \xdef\i{0}
    \xdef\x{0}
    \xdef\y{0}
    \xdef\corner{3}
    \xdef\sidelen{0}
    \xdef\nextsidelen{1}
    \xdef\direction{3}
    \incr\maxnums
    \incr\maxcoordsX
    \incr\maxcoordsY
    
    \foreach \x in {0,...,\maxcoordsX} {
      \foreach \y in {1,...,\maxcoordsY} {
        \expandafter\xdef\csname dotxypresent-\x-\y\endcsname{0}
      }
    }
    
    \def\reallysmallcheat#1{%
      \ensuremath{\hphantom{\vphantom{X}}_{\hphantom{\vphantom{X}}_{#1}}}
    }
    \def\drawnode{%
      \node[dot] (dot-\x-\y) at (\x,-\y) {};
      \node[dotphantom] (phantom-\x-\y) at (\x,-\y) {};
      \ifnum\maxnums>\i
        % \ifnum\corner=0    \node[numero,anchor=south]      at (dot-\x-\y.north)      {\i}; \else
        % \ifnum\corner=1    \node[numero,anchor=east]       at (dot-\x-\y.west)       {\i}; \else
        % \ifnum\corner=2    \node[numero,anchor=east]       at (dot-\x-\y.west)       {\i}; \else
        % \ifnum\corner=3    \node[numero,anchor=south]      at (dot-\x-\y.north)      {\i}; \else
        % \ifnum\direction=0 \node[numero,anchor=north west] at (dot-\x-\y.south east) {\i}; \else
        % \ifnum\direction=2 \node[numero,anchor=north west] at (dot-\x-\y.south east) {\i}; \fi\fi\fi\fi\fi\fi
        \node[numero,anchor=north west] at (dot-\x-\y.south east) {\i};
      \fi
      \ifnum\maxcoordsX>\x
      \ifnum\maxcoordsY>\y
        % \ifnum\corner=0    \node[coord,anchor=south east] at (dot-\x-\y.north west) {\reallysmallcheat{(\x, \y)}}; \else
        % \ifnum\corner=1    \node[coord,anchor=north east] at (dot-\x-\y.south west) {\reallysmallcheat{(\x, \y)}}; \else
        % \ifnum\corner=2    \node[coord,anchor=north west] at (dot-\x-\y.south east) {\reallysmallcheat{(\x, \y)}}; \else
        % \ifnum\corner=3    \node[coord,anchor=south west] at (dot-\x-\y.north east) {\reallysmallcheat{(\x, \y)}}; \else
        % \ifnum\direction=0 \node[coord,anchor=east]       at (dot-\x-\y.west)       {\reallysmallcheat{(\x, \y)}}; \else
        % \ifnum\direction=1 \node[coord,anchor=north]      at (dot-\x-\y.south)      {\reallysmallcheat{(\x, \y)}}; \else
        % \ifnum\direction=2 \node[coord,anchor=west]       at (dot-\x-\y.east)       {\reallysmallcheat{(\x, \y)}}; \else
        % \ifnum\direction=3 \node[coord,anchor=south]      at (dot-\x-\y.north)      {\reallysmallcheat{(\x, \y)}}; \fi\fi\fi\fi\fi\fi\fi\fi
        \expandafter\xdef\csname dotxypresent-\x-\y\endcsname{1}
        \ifnum\y=0
          \node[coord,anchor=south west] at (dot-\x-\y.north east) {\reallysmallcheat{(\x, \y)}};
        \else
          \node[coord,anchor=west,xshift=0.2] at (dot-\x-\y.east) {\reallysmallcheat{(\x, \y)}};
        \fi
      \fi
      \fi
    }
    \def\drawlinknode{%
      \drawnode%
      \draw[->] (phantom-\oldx-\oldy) -- (phantom-\x-\y);%
    }
    \def\mystep{%
      \incr\i%
      \xdef\oldx{\x}%
      \xdef\oldy{\y}%
    }
    \def\changedir#1{%
      \xdef\direction{#1}%
      \xdef\corner{#1}%
    }
    
    \drawnode
    \mystep \incr\x
    \xdef\cornerc{0}

    \foreach \pos in {1,...,\maxdots} {
      % Detect when we are at the end of this edge.
      \ifnum\sidelen=0
        \ifnum\direction=0 \changedir{1} \incr\nextsidelen \xdef\sidelen{1}            \else
        \ifnum\direction=1 \changedir{2}                   \xdef\sidelen{\nextsidelen} \else
        \ifnum\direction=2 \changedir{3} \incr\nextsidelen \xdef\sidelen{1}            \else
        \ifnum\direction=3 \changedir{0}                   \xdef\sidelen{\nextsidelen} \fi\fi\fi\fi% Brin d'acier
        
      \fi
      % Draw node and link to previous, step counters
      \drawlinknode \mystep
      %% Se déplacer vers ↙↓↗→
      \ifnum\direction=0 \incr\y \decr\x \fi
      \ifnum\direction=1 \incr\y         \fi
      \ifnum\direction=2 \decr\y \incr\x \fi
      \ifnum\direction=3         \incr\x \fi
      \decr\sidelen
      \xdef\corner{4}%% 4 == pas de coin-coin.
    }
    
    \foreach \x in {0,...,\maxdotsX} {
      \foreach \y in {1,...,\maxdotsY} {
        \node[dot] (dot-\x-\y) at (\x,-\y) {};
      }
    }
    \decr\maxcoordsX
    \decr\maxcoordsY
    \foreach \x in {0,...,\maxcoordsX} {
      \foreach \y in {0,...,\maxcoordsY} {
        \node[dot,fill=none] (fakedot-\x-\y) at (\x,-\y) {};
        \expandafter\ifnum\csname dotxypresent-\x-\y\endcsname=0
          \ifnum\y=0
            \node[coord,anchor=south west] at (fakedot-\x-\y.north east) {\reallysmallcheat{(\x, \y)}};
          \else
            \node[coord,anchor=west,xshift=0.2] at (fakedot-\x-\y.east) {\reallysmallcheat{(\x, \y)}};
          \fi
        \fi
      }
    }
    
    % \foreach \i in {1, ...,6}{
    %   \node[constatation1] at (.5*\i,-.5*\i) {\i};
    % }
    % \node[constatation1] at (.5*7,-.5*7) {\vdots};
    
    % \node[coordinate] (fake0-0) at (0,0) {};
    % \node[coordinate] (fake1-0) at (1,0) {};
    % \node[coordinate] (fake3-0) at (3,0) {};
    % \node[coordinate] (fake5-0) at (5,0) {};
    % \node[coordinate] (fake0-2) at (0,-2) {};
    % \node[coordinate] (fake0-4) at (0,-4) {};
    % \node[coordinate] (fake0-6) at (0,-6) {};
    % \node[constatation2, above left of=fake0-0] {0};
    % \node[constatation2, above of=fake1-0]  {1};
    % \node[constatation2, above of=fake3-0]  {6};
    % \node[constatation2, above of=fake5-0] {15};
    % \node[constatation2, left of=fake0-2] {3};
    % \node[constatation2, left of=fake0-4] {10};
    % \node[constatation2, left of=fake0-6] {21};
  \end{tikzpicture}
  \caption{Codage d'un couple d'entiers relatifs}%% TODO : caption
  \label{fig:codage-nat-sans-constatation}
\end{figure}

% \begin{figure}[h!]
%   \centering
%   \begin{tikzpicture}[
%       dot/.style = {
%         circle,
%         fill=black,
%         inner sep=0.5pt
%       },
%       arc/.style = {
%         ->,
%         >=stealth
%       }
%     ]
%     \foreach \xpos in {0, ..., 4} {
%       \foreach \ypos in {0, ..., 4} {
%         \node[dot] at (\xpos,-\ypos) {};
%       }
%     }
    
%     \draw[arc] (0,-0) -- (1,-0);
    
%     \draw[arc] (1,-0) -- (0,-1);
%     \draw[arc] (0,-1) -- (0,-2);
    
%     \draw[arc] (0,-2) -- (1,-1);
%     \draw[arc] (1,-1) -- (2,-0);
%     \draw[arc] (2,-0) -- (3,-0);
    
%     \draw[arc] (3,-0) -- (2,-1);
%     \draw[arc] (2,-1) -- (1,-2);
%     \draw[arc] (1,-2) -- (0,-3);
%     \draw[arc] (0,-3) -- (0,-4);
    
%     \draw[arc] (0,-4) -- (1,-3);
%     \draw[arc,dashed] (1,-3) -- (2,-2);
%   \end{tikzpicture}
%   \caption{Codage d'un couple d'entiers relatifs}
%   \label{fig:codage-rel}
% \end{figure}

\end{document}