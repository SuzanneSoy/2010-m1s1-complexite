\documentclass{article}

\usepackage[utf8]{inputenc}
\usepackage[T1]{fontenc}
\usepackage[frenchb]{babel}
\usepackage{tikz}
\usepackage{amsmath}
\usepackage{listings}
\usepackage{amssymb}
\usetikzlibrary{chains,positioning,matrix,arrows,decorations,calc}
\title{Rapport de projet : FMIN105\\ Cours algorithmique / complexité / calculabilité}
\author{\textsc{Bonavero} Yoann \\ \textsc{Brun} Bertrand \\ \textsc{Charron} John \\ \textsc{Dupéron} Georges}
\date{}

\setlength{\parindent}{0pt}
\setlength{\parskip}{2ex}

\newcounter{exocount}
\setcounter{exocount}{0}

\newcounter{enoncecount}
\setcounter{enoncecount}{0}

\newenvironment{enonce}
{
\stepcounter{enoncecount}
\bf\small \arabic{enoncecount}.
\begin{bf}
}
{
\end{bf}
}


\newcounter{sousenoncecount}
\setcounter{sousenoncecount}{0}
\newenvironment{sousenonce}
{
\stepcounter{sousenoncecount}
\bf\small (\alph{sousenoncecount})
\begin{bf}
}
{
\end{bf}
}

\begin{document}
\setcounter{page}{0}
\pagestyle{plain}

\makeatletter
\def\myadv#1#2{%
  \xdef\incr@backup{\the\@tempcnta}%
  \@tempcnta=#1{}%
  \advance\@tempcnta by #2{}%
  \xdef#1{\the\@tempcnta}%
  \@tempcnta=\incr@backup{}%
}
\def\incr#1{%
  \myadv#1{1}%
}
\def\decr#1{%
  \myadv#1{-1}%
}
\makeatother

\begin{figure}[h!]
  \xdef\maxdots{81}
  \xdef\maxnums{56}
  \xdef\maxcoords{20}
  \centering
  \begin{tikzpicture}[
    dot/.style={
      circle,
      inner sep=0.5pt,
      outer sep=0.5pt,
      fill=black
    },
    numero/.style={red},
    ]
    \xdef\i{0}
    \xdef\x{0}
    \xdef\y{0}
    \xdef\corner{0}
    \xdef\sidelen{1}
    \xdef\nextsidelen{1}
    \xdef\direction{0}
    \incr\maxnums
    \incr\maxcoords
    
    \def\reallysmallcheat#1{%
      \ensuremath{\hphantom{\vphantom{X}}_{\hphantom{\vphantom{X}}_{#1}}}
    }
    \def\drawnode{%
      \node[dot] (\x-\y) at (\x,\y) {};
      \ifnum\maxnums>\i
        \ifnum\corner=0    \node[numero,anchor=north west] at (\x-\y.south east) {\i}; \else
        \ifnum\corner=1    \node[numero,anchor=south west] at (\x-\y.north east) {\i}; \else
        \ifnum\corner=2    \node[numero,anchor=south east] at (\x-\y.north west) {\i}; \else
        \ifnum\corner=3    \node[numero,anchor=north east] at (\x-\y.south west) {\i}; \else
        \ifnum\direction=0 \node[numero,anchor=west]       at (\x-\y.east)       {\i}; \else
        \ifnum\direction=1 \node[numero,anchor=south]      at (\x-\y.north)      {\i}; \else
        \ifnum\direction=2 \node[numero,anchor=east]       at (\x-\y.west)       {\i}; \else
        \ifnum\direction=3 \node[numero,anchor=north]      at (\x-\y.south)      {\i}; \fi\fi\fi\fi\fi\fi\fi\fi
      \fi
      % {\i\reallysmallcheat{(\x, \y)}};
    }
    \def\drawlinknode{%
      \drawnode%
      \draw[->] (\oldx-\oldy) -- (\x-\y);%
    }
    \def\mystep{%
      \incr\i%
      \xdef\oldx{\x}%
      \xdef\oldy{\y}%
    }
    
    \drawnode
    \mystep \incr\x

    \foreach \pos in {1,...,\maxdots} {
      % Detect when we are at the end of this edge.
      \ifnum\sidelen=0
        \ifnum\direction=0 \xdef\direction{1} \xdef\corner{1} \incr\nextsidelen \else
        \ifnum\direction=1 \xdef\direction{2} \xdef\corner{2}                   \else
        \ifnum\direction=2 \xdef\direction{3} \xdef\corner{3} \incr\nextsidelen \else
        \ifnum\direction=3 \xdef\direction{0} \xdef\corner{0}                   \fi\fi\fi\fi% Brin d'acier
        \xdef\sidelen{\nextsidelen}
      \fi
      % Draw node and link to previous, step counters
      \drawlinknode \mystep
      %% Se déplacer vers ↑←↓→
      \ifnum\direction=0 \incr\y \fi
      \ifnum\direction=1 \decr\x \fi
      \ifnum\direction=2 \decr\y \fi
      \ifnum\direction=3 \incr\x \fi
      \decr\sidelen
      \xdef\corner{4}%% 4 == pas de coin-coin.
    }
    % \foreach \sidelen in {2,4, ...,6} {
    %   \foreach \pos in {2, ..., \sidelen} { \drawlinknode \mystep \incr\y }
    %   \foreach \pos in {1, ..., \sidelen} { \drawlinknode \mystep \decr\x }
    %   \foreach \pos in {1, ..., \sidelen} { \drawlinknode \mystep \decr\y }
    %   \foreach \pos in {0, ..., \sidelen} { \drawlinknode \mystep \incr\x }
    % }
    % %% Dernier côté :
    % \foreach \pos in {2, ..., 8} { \drawlinknode \mystep \incr\y }
  \end{tikzpicture}
  \caption{TODO : caption}
  \label{fig:graphe-g}
\end{figure}

\end{document}