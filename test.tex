\documentclass{article}

\usepackage[utf8]{inputenc}
\usepackage[T1]{fontenc}
\usepackage[frenchb]{babel}
\usepackage{tikz}
\usepackage{amsmath}
\usepackage{listings}
\usepackage{amssymb}
\usetikzlibrary{chains,positioning,matrix,arrows,decorations,calc}
\title{Rapport de projet : FMIN105\\ Cours algorithmique / complexité / calculabilité}
\author{\textsc{Bonavero} Yoann \\ \textsc{Brun} Bertrand \\ \textsc{Charron} John \\ \textsc{Dupéron} Georges}
\date{}

\setlength{\parindent}{0pt}
\setlength{\parskip}{2ex}

\newcounter{exocount}
\setcounter{exocount}{0}

\newcounter{enoncecount}
\setcounter{enoncecount}{0}

\newenvironment{enonce}
{
\stepcounter{enoncecount}
\bf\small \arabic{enoncecount}.
\begin{bf}
}
{
\end{bf}
}


\newcounter{sousenoncecount}
\setcounter{sousenoncecount}{0}
\newenvironment{sousenonce}
{
\stepcounter{sousenoncecount}
\bf\small (\alph{sousenoncecount})
\begin{bf}
}
{
\end{bf}
}

\begin{document}
\setcounter{page}{0}
\pagestyle{plain}

\begin{figure}[h!]
  %% «Paramètres»
  \xdef\maxdots{81}
  \xdef\maxnums{56}
  \xdef\maxcoords{20}
  
  %% ===== Code is below =====
  
  %% Macros crades
  \makeatletter
  \def\myadv#1#2{%
    \xdef\incr@backup{\the\@tempcnta}%
    \@tempcnta=#1{}%
    \advance\@tempcnta by #2{}%
    \xdef#1{\the\@tempcnta}%
    \@tempcnta=\incr@backup{}%
  }
  \def\incr#1{%
    \myadv#1{1}%
  }
  \def\decr#1{%
    \myadv#1{-1}%
  }
  \makeatother
  % \centering
  \hskip -21mm%% Hack-o-matic to get the picture more or less centered…
  \begin{tikzpicture}[
    dot/.style={
      circle,
      inner sep=0.7pt,
      fill=black
    },
    dotphantom/.style={
      circle,
      inner sep=3pt,
      fill=none,
      draw=none
    },
    numero/.style={circle, fill=white, inner sep=0.2pt, text=red, scale=.75},
    coord/.style={darkgray, scale=.6},
    %
    bigarrow/.style={
      ->,
      thick,
      draw=green!50!black
    }
    ]
    
    % Diagonales
    \node[dotphantom] (pre-phantom-0-0) at (0,0) {};
    \node[dotphantom] (pre-phantom-1-0) at (1,0) {};
    
    % Top right
    \node[inner sep=0.1pt] (tr-start) at (4,4) {};
    \xdef\previous{tr-start}
    \foreach \pos/\content in {0/$1\times 2=2$,1/$3\times 4=12$,2/$5\times 6=30$,3/$7\times 8=56$,4/$\vphantom{X}\dots$}{
      \node[anchor=south] (tr-temp-\pos) at (intersection cs: first line={(\previous.north west)--(\previous.north east)}, second line={(0,0)--(6,6)}) {\phantom{\content}};
      \node[anchor=north west] (tr-\pos) at (intersection cs: first line={(tr-temp-\pos.north west)--(tr-temp-\pos.north east)}, second line={(0,0)--(6,6)}) {\content};
      \xdef\previous{tr-\pos}
    }
    \draw[bigarrow] (pre-phantom-0-0) -- (intersection cs: first line={(\previous.north west)--(\previous.south west)}, second line={(0,0)--(6,6)});
    
    % Top left
    \node[inner sep=0.1pt] (tl-start) at (-4,4) {};
    \xdef\previous{tl-start}
    \foreach \pos/\content in {0/$0^2=0$,1/$2^2=4$,2/$6^2=36$,3/$8^2=64$,4/$\vphantom{X}\dots$}{
      \node[anchor=south] (tl-temp-\pos) at (intersection cs: first line={(\previous.north west)--(\previous.north east)}, second line={(0,0)--(-6,6)}) {\phantom{\content}};
      \node[anchor=north east] (tl-\pos) at (intersection cs: first line={(tl-temp-\pos.north west)--(tl-temp-\pos.north east)}, second line={(0,0)--(-6,6)}) {\content};
      \xdef\previous{tl-\pos}
    }
    \draw[bigarrow] (pre-phantom-0-0) -- (intersection cs: first line={(\previous.north east)--(\previous.south east)}, second line={(0,0)--(-6,6)});
    
    % Bottom left
    \node[inner sep=0.1pt] (bl-start) at (-4,-4) {};
    \xdef\previous{bl-start}
    \foreach \pos/\content in {0/$0\times 1=0$,1/$2\times 3=6$,2/$4\times 5=20$,3/$6\times 7=42$,4/$\vphantom{X}\dots$}{
      \node[anchor=north] (bl-temp-\pos) at (intersection cs: first line={(\previous.south west)--(\previous.south east)}, second line={(0,0)--(-6,-6)}) {\phantom{\content}};
      \node[anchor=south east] (bl-\pos) at (intersection cs: first line={(bl-temp-\pos.south west)--(bl-temp-\pos.south east)}, second line={(0,0)--(-6,-6)}) {\content};
      \xdef\previous{bl-\pos}
    }
    \draw[bigarrow] (pre-phantom-0-0) -- (intersection cs: first line={(\previous.north east)--(\previous.south east)}, second line={(0,0)--(-6,-6)});
    
    % Bottom right
    \node[inner sep=0.1pt] (br-start) at (5,-4) {};
    \xdef\previous{br-start}
    \foreach \pos/\content in {0/$1^2=1$,1/$3^2=9$,2/$5^2=25$,3/$7^2=49$,4/$\vphantom{X}\dots$}{
      \node[anchor=north] (br-temp-\pos) at (intersection cs: first line={(\previous.south west)--(\previous.south east)}, second line={(1,0)--(6,-5)}) {\phantom{\content}};
      \node[anchor=south west] (br-\pos) at (intersection cs: first line={(br-temp-\pos.south west)--(br-temp-\pos.south east)}, second line={(1,0)--(6,-5)}) {\content};
      \xdef\previous{br-\pos}
    }
    \draw[bigarrow] (pre-phantom-1-0) -- (intersection cs: first line={(\previous.north west)--(\previous.south west)}, second line={(1,0)--(6,-5)});
    % Fin diagonales

    % Définitions pour le code ci-dessous.
    \xdef\i{0}
    \xdef\x{0}
    \xdef\y{0}
    \xdef\corner{4}
    \xdef\sidelen{0}
    \xdef\nextsidelen{1}
    \xdef\direction{3}
    \incr\maxnums
    \incr\maxcoords
    
    \def\reallysmallcheat#1{%
      \ensuremath{\hphantom{\vphantom{X}}_{\hphantom{\vphantom{X}}_{#1}}}
    }
    \def\drawnode{%
      \node[dot] (dot-\x-\y) at (\x,\y) {};
      \node[dotphantom] (phantom-\x-\y) at (\x,\y) {};
      \ifnum\maxnums>\i
        \ifnum\corner=0    \node[numero,anchor=north west] at (dot-\x-\y.south east) {\i}; \else
        \ifnum\corner=1    \node[numero,anchor=south west] at (dot-\x-\y.north east) {\i}; \else
        \ifnum\corner=2    \node[numero,anchor=south east] at (dot-\x-\y.north west) {\i}; \else
        \ifnum\corner=3    \node[numero,anchor=north east] at (dot-\x-\y.south west) {\i}; \else
        \ifnum\direction=0 \node[numero,anchor=west]       at (dot-\x-\y.east)       {\i}; \else
        \ifnum\direction=1 \node[numero,anchor=south]      at (dot-\x-\y.north)      {\i}; \else
        \ifnum\direction=2 \node[numero,anchor=east]       at (dot-\x-\y.west)       {\i}; \else
        \ifnum\direction=3 \node[numero,anchor=north]      at (dot-\x-\y.south)      {\i}; \fi\fi\fi\fi\fi\fi\fi\fi
      \fi
      \ifnum\maxcoords>\i
        \ifnum\corner=0    \node[coord,anchor=south east] at (dot-\x-\y.north west) {\reallysmallcheat{(\x, \y)}}; \else
        \ifnum\corner=1    \node[coord,anchor=north east] at (dot-\x-\y.south west) {\reallysmallcheat{(\x, \y)}}; \else
        \ifnum\corner=2    \node[coord,anchor=north west] at (dot-\x-\y.south east) {\reallysmallcheat{(\x, \y)}}; \else
        \ifnum\corner=3    \node[coord,anchor=south west] at (dot-\x-\y.north east) {\reallysmallcheat{(\x, \y)}}; \else
        \ifnum\direction=0 \node[coord,anchor=east]       at (dot-\x-\y.west)       {\reallysmallcheat{(\x, \y)}}; \else
        \ifnum\direction=1 \node[coord,anchor=north]      at (dot-\x-\y.south)      {\reallysmallcheat{(\x, \y)}}; \else
        \ifnum\direction=2 \node[coord,anchor=west]       at (dot-\x-\y.east)       {\reallysmallcheat{(\x, \y)}}; \else
        \ifnum\direction=3 \node[coord,anchor=south]      at (dot-\x-\y.north)      {\reallysmallcheat{(\x, \y)}}; \fi\fi\fi\fi\fi\fi\fi\fi
      \fi
    }
    \def\drawlinknode{%
      \drawnode%
      \draw[->] (phantom-\oldx-\oldy) -- (phantom-\x-\y);%
    }
    \def\mystep{%
      \incr\i%
      \xdef\oldx{\x}%
      \xdef\oldy{\y}%
    }
    \def\changedir#1{%
      \xdef\direction{#1}%
      \xdef\corner{#1}%
    }
    
    \drawnode
    \mystep \incr\x
    \xdef\cornerc{0}

    \foreach \pos in {1,...,\maxdots} {
      % Detect when we are at the end of this edge.
      \ifnum\sidelen=0
        \ifnum\direction=0 \changedir{1} \incr\nextsidelen \else
        \ifnum\direction=1 \changedir{2}                   \else
        \ifnum\direction=2 \changedir{3} \incr\nextsidelen \else
        \ifnum\direction=3 \changedir{0}                   \fi\fi\fi\fi% Brin d'acier
        \xdef\sidelen{\nextsidelen}
      \fi
      % Draw node and link to previous, step counters
      \drawlinknode \mystep
      %% Se déplacer vers ↑←↓→
      \ifnum\direction=0 \incr\y \fi
      \ifnum\direction=1 \decr\x \fi
      \ifnum\direction=2 \decr\y \fi
      \ifnum\direction=3 \incr\x \fi
      \decr\sidelen
      \xdef\corner{4}%% 4 == pas de coin-coin.
    }
  \end{tikzpicture}
  \caption{Codage d'un couple d'entiers relatifs}
  \label{fig:codage-rel}
\end{figure}

\end{document}